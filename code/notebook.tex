
% Default to the notebook output style

    


% Inherit from the specified cell style.




    
\documentclass[11pt]{article}

    
    
    \usepackage[T1]{fontenc}
    % Nicer default font (+ math font) than Computer Modern for most use cases
    \usepackage{mathpazo}

    % Basic figure setup, for now with no caption control since it's done
    % automatically by Pandoc (which extracts ![](path) syntax from Markdown).
    \usepackage{graphicx}
    % We will generate all images so they have a width \maxwidth. This means
    % that they will get their normal width if they fit onto the page, but
    % are scaled down if they would overflow the margins.
    \makeatletter
    \def\maxwidth{\ifdim\Gin@nat@width>\linewidth\linewidth
    \else\Gin@nat@width\fi}
    \makeatother
    \let\Oldincludegraphics\includegraphics
    % Set max figure width to be 80% of text width, for now hardcoded.
    \renewcommand{\includegraphics}[1]{\Oldincludegraphics[width=.8\maxwidth]{#1}}
    % Ensure that by default, figures have no caption (until we provide a
    % proper Figure object with a Caption API and a way to capture that
    % in the conversion process - todo).
    \usepackage{caption}
    \DeclareCaptionLabelFormat{nolabel}{}
    \captionsetup{labelformat=nolabel}

    \usepackage{adjustbox} % Used to constrain images to a maximum size 
    \usepackage{xcolor} % Allow colors to be defined
    \usepackage{enumerate} % Needed for markdown enumerations to work
    \usepackage{geometry} % Used to adjust the document margins
    \usepackage{amsmath} % Equations
    \usepackage{amssymb} % Equations
    \usepackage{textcomp} % defines textquotesingle
    % Hack from http://tex.stackexchange.com/a/47451/13684:
    \AtBeginDocument{%
        \def\PYZsq{\textquotesingle}% Upright quotes in Pygmentized code
    }
    \usepackage{upquote} % Upright quotes for verbatim code
    \usepackage{eurosym} % defines \euro
    \usepackage[mathletters]{ucs} % Extended unicode (utf-8) support
    \usepackage[utf8x]{inputenc} % Allow utf-8 characters in the tex document
    \usepackage{fancyvrb} % verbatim replacement that allows latex
    \usepackage{grffile} % extends the file name processing of package graphics 
                         % to support a larger range 
    % The hyperref package gives us a pdf with properly built
    % internal navigation ('pdf bookmarks' for the table of contents,
    % internal cross-reference links, web links for URLs, etc.)
    \usepackage{hyperref}
    \usepackage{longtable} % longtable support required by pandoc >1.10
    \usepackage{booktabs}  % table support for pandoc > 1.12.2
    \usepackage[inline]{enumitem} % IRkernel/repr support (it uses the enumerate* environment)
    \usepackage[normalem]{ulem} % ulem is needed to support strikethroughs (\sout)
                                % normalem makes italics be italics, not underlines
    

    
    
    % Colors for the hyperref package
    \definecolor{urlcolor}{rgb}{0,.145,.698}
    \definecolor{linkcolor}{rgb}{.71,0.21,0.01}
    \definecolor{citecolor}{rgb}{.12,.54,.11}

    % ANSI colors
    \definecolor{ansi-black}{HTML}{3E424D}
    \definecolor{ansi-black-intense}{HTML}{282C36}
    \definecolor{ansi-red}{HTML}{E75C58}
    \definecolor{ansi-red-intense}{HTML}{B22B31}
    \definecolor{ansi-green}{HTML}{00A250}
    \definecolor{ansi-green-intense}{HTML}{007427}
    \definecolor{ansi-yellow}{HTML}{DDB62B}
    \definecolor{ansi-yellow-intense}{HTML}{B27D12}
    \definecolor{ansi-blue}{HTML}{208FFB}
    \definecolor{ansi-blue-intense}{HTML}{0065CA}
    \definecolor{ansi-magenta}{HTML}{D160C4}
    \definecolor{ansi-magenta-intense}{HTML}{A03196}
    \definecolor{ansi-cyan}{HTML}{60C6C8}
    \definecolor{ansi-cyan-intense}{HTML}{258F8F}
    \definecolor{ansi-white}{HTML}{C5C1B4}
    \definecolor{ansi-white-intense}{HTML}{A1A6B2}

    % commands and environments needed by pandoc snippets
    % extracted from the output of `pandoc -s`
    \providecommand{\tightlist}{%
      \setlength{\itemsep}{0pt}\setlength{\parskip}{0pt}}
    \DefineVerbatimEnvironment{Highlighting}{Verbatim}{commandchars=\\\{\}}
    % Add ',fontsize=\small' for more characters per line
    \newenvironment{Shaded}{}{}
    \newcommand{\KeywordTok}[1]{\textcolor[rgb]{0.00,0.44,0.13}{\textbf{{#1}}}}
    \newcommand{\DataTypeTok}[1]{\textcolor[rgb]{0.56,0.13,0.00}{{#1}}}
    \newcommand{\DecValTok}[1]{\textcolor[rgb]{0.25,0.63,0.44}{{#1}}}
    \newcommand{\BaseNTok}[1]{\textcolor[rgb]{0.25,0.63,0.44}{{#1}}}
    \newcommand{\FloatTok}[1]{\textcolor[rgb]{0.25,0.63,0.44}{{#1}}}
    \newcommand{\CharTok}[1]{\textcolor[rgb]{0.25,0.44,0.63}{{#1}}}
    \newcommand{\StringTok}[1]{\textcolor[rgb]{0.25,0.44,0.63}{{#1}}}
    \newcommand{\CommentTok}[1]{\textcolor[rgb]{0.38,0.63,0.69}{\textit{{#1}}}}
    \newcommand{\OtherTok}[1]{\textcolor[rgb]{0.00,0.44,0.13}{{#1}}}
    \newcommand{\AlertTok}[1]{\textcolor[rgb]{1.00,0.00,0.00}{\textbf{{#1}}}}
    \newcommand{\FunctionTok}[1]{\textcolor[rgb]{0.02,0.16,0.49}{{#1}}}
    \newcommand{\RegionMarkerTok}[1]{{#1}}
    \newcommand{\ErrorTok}[1]{\textcolor[rgb]{1.00,0.00,0.00}{\textbf{{#1}}}}
    \newcommand{\NormalTok}[1]{{#1}}
    
    % Additional commands for more recent versions of Pandoc
    \newcommand{\ConstantTok}[1]{\textcolor[rgb]{0.53,0.00,0.00}{{#1}}}
    \newcommand{\SpecialCharTok}[1]{\textcolor[rgb]{0.25,0.44,0.63}{{#1}}}
    \newcommand{\VerbatimStringTok}[1]{\textcolor[rgb]{0.25,0.44,0.63}{{#1}}}
    \newcommand{\SpecialStringTok}[1]{\textcolor[rgb]{0.73,0.40,0.53}{{#1}}}
    \newcommand{\ImportTok}[1]{{#1}}
    \newcommand{\DocumentationTok}[1]{\textcolor[rgb]{0.73,0.13,0.13}{\textit{{#1}}}}
    \newcommand{\AnnotationTok}[1]{\textcolor[rgb]{0.38,0.63,0.69}{\textbf{\textit{{#1}}}}}
    \newcommand{\CommentVarTok}[1]{\textcolor[rgb]{0.38,0.63,0.69}{\textbf{\textit{{#1}}}}}
    \newcommand{\VariableTok}[1]{\textcolor[rgb]{0.10,0.09,0.49}{{#1}}}
    \newcommand{\ControlFlowTok}[1]{\textcolor[rgb]{0.00,0.44,0.13}{\textbf{{#1}}}}
    \newcommand{\OperatorTok}[1]{\textcolor[rgb]{0.40,0.40,0.40}{{#1}}}
    \newcommand{\BuiltInTok}[1]{{#1}}
    \newcommand{\ExtensionTok}[1]{{#1}}
    \newcommand{\PreprocessorTok}[1]{\textcolor[rgb]{0.74,0.48,0.00}{{#1}}}
    \newcommand{\AttributeTok}[1]{\textcolor[rgb]{0.49,0.56,0.16}{{#1}}}
    \newcommand{\InformationTok}[1]{\textcolor[rgb]{0.38,0.63,0.69}{\textbf{\textit{{#1}}}}}
    \newcommand{\WarningTok}[1]{\textcolor[rgb]{0.38,0.63,0.69}{\textbf{\textit{{#1}}}}}
    
    
    % Define a nice break command that doesn't care if a line doesn't already
    % exist.
    \def\br{\hspace*{\fill} \\* }
    % Math Jax compatability definitions
    \def\gt{>}
    \def\lt{<}
    % Document parameters
    \title{chap04}
    
    
    

    % Pygments definitions
    
\makeatletter
\def\PY@reset{\let\PY@it=\relax \let\PY@bf=\relax%
    \let\PY@ul=\relax \let\PY@tc=\relax%
    \let\PY@bc=\relax \let\PY@ff=\relax}
\def\PY@tok#1{\csname PY@tok@#1\endcsname}
\def\PY@toks#1+{\ifx\relax#1\empty\else%
    \PY@tok{#1}\expandafter\PY@toks\fi}
\def\PY@do#1{\PY@bc{\PY@tc{\PY@ul{%
    \PY@it{\PY@bf{\PY@ff{#1}}}}}}}
\def\PY#1#2{\PY@reset\PY@toks#1+\relax+\PY@do{#2}}

\expandafter\def\csname PY@tok@w\endcsname{\def\PY@tc##1{\textcolor[rgb]{0.73,0.73,0.73}{##1}}}
\expandafter\def\csname PY@tok@c\endcsname{\let\PY@it=\textit\def\PY@tc##1{\textcolor[rgb]{0.25,0.50,0.50}{##1}}}
\expandafter\def\csname PY@tok@cp\endcsname{\def\PY@tc##1{\textcolor[rgb]{0.74,0.48,0.00}{##1}}}
\expandafter\def\csname PY@tok@k\endcsname{\let\PY@bf=\textbf\def\PY@tc##1{\textcolor[rgb]{0.00,0.50,0.00}{##1}}}
\expandafter\def\csname PY@tok@kp\endcsname{\def\PY@tc##1{\textcolor[rgb]{0.00,0.50,0.00}{##1}}}
\expandafter\def\csname PY@tok@kt\endcsname{\def\PY@tc##1{\textcolor[rgb]{0.69,0.00,0.25}{##1}}}
\expandafter\def\csname PY@tok@o\endcsname{\def\PY@tc##1{\textcolor[rgb]{0.40,0.40,0.40}{##1}}}
\expandafter\def\csname PY@tok@ow\endcsname{\let\PY@bf=\textbf\def\PY@tc##1{\textcolor[rgb]{0.67,0.13,1.00}{##1}}}
\expandafter\def\csname PY@tok@nb\endcsname{\def\PY@tc##1{\textcolor[rgb]{0.00,0.50,0.00}{##1}}}
\expandafter\def\csname PY@tok@nf\endcsname{\def\PY@tc##1{\textcolor[rgb]{0.00,0.00,1.00}{##1}}}
\expandafter\def\csname PY@tok@nc\endcsname{\let\PY@bf=\textbf\def\PY@tc##1{\textcolor[rgb]{0.00,0.00,1.00}{##1}}}
\expandafter\def\csname PY@tok@nn\endcsname{\let\PY@bf=\textbf\def\PY@tc##1{\textcolor[rgb]{0.00,0.00,1.00}{##1}}}
\expandafter\def\csname PY@tok@ne\endcsname{\let\PY@bf=\textbf\def\PY@tc##1{\textcolor[rgb]{0.82,0.25,0.23}{##1}}}
\expandafter\def\csname PY@tok@nv\endcsname{\def\PY@tc##1{\textcolor[rgb]{0.10,0.09,0.49}{##1}}}
\expandafter\def\csname PY@tok@no\endcsname{\def\PY@tc##1{\textcolor[rgb]{0.53,0.00,0.00}{##1}}}
\expandafter\def\csname PY@tok@nl\endcsname{\def\PY@tc##1{\textcolor[rgb]{0.63,0.63,0.00}{##1}}}
\expandafter\def\csname PY@tok@ni\endcsname{\let\PY@bf=\textbf\def\PY@tc##1{\textcolor[rgb]{0.60,0.60,0.60}{##1}}}
\expandafter\def\csname PY@tok@na\endcsname{\def\PY@tc##1{\textcolor[rgb]{0.49,0.56,0.16}{##1}}}
\expandafter\def\csname PY@tok@nt\endcsname{\let\PY@bf=\textbf\def\PY@tc##1{\textcolor[rgb]{0.00,0.50,0.00}{##1}}}
\expandafter\def\csname PY@tok@nd\endcsname{\def\PY@tc##1{\textcolor[rgb]{0.67,0.13,1.00}{##1}}}
\expandafter\def\csname PY@tok@s\endcsname{\def\PY@tc##1{\textcolor[rgb]{0.73,0.13,0.13}{##1}}}
\expandafter\def\csname PY@tok@sd\endcsname{\let\PY@it=\textit\def\PY@tc##1{\textcolor[rgb]{0.73,0.13,0.13}{##1}}}
\expandafter\def\csname PY@tok@si\endcsname{\let\PY@bf=\textbf\def\PY@tc##1{\textcolor[rgb]{0.73,0.40,0.53}{##1}}}
\expandafter\def\csname PY@tok@se\endcsname{\let\PY@bf=\textbf\def\PY@tc##1{\textcolor[rgb]{0.73,0.40,0.13}{##1}}}
\expandafter\def\csname PY@tok@sr\endcsname{\def\PY@tc##1{\textcolor[rgb]{0.73,0.40,0.53}{##1}}}
\expandafter\def\csname PY@tok@ss\endcsname{\def\PY@tc##1{\textcolor[rgb]{0.10,0.09,0.49}{##1}}}
\expandafter\def\csname PY@tok@sx\endcsname{\def\PY@tc##1{\textcolor[rgb]{0.00,0.50,0.00}{##1}}}
\expandafter\def\csname PY@tok@m\endcsname{\def\PY@tc##1{\textcolor[rgb]{0.40,0.40,0.40}{##1}}}
\expandafter\def\csname PY@tok@gh\endcsname{\let\PY@bf=\textbf\def\PY@tc##1{\textcolor[rgb]{0.00,0.00,0.50}{##1}}}
\expandafter\def\csname PY@tok@gu\endcsname{\let\PY@bf=\textbf\def\PY@tc##1{\textcolor[rgb]{0.50,0.00,0.50}{##1}}}
\expandafter\def\csname PY@tok@gd\endcsname{\def\PY@tc##1{\textcolor[rgb]{0.63,0.00,0.00}{##1}}}
\expandafter\def\csname PY@tok@gi\endcsname{\def\PY@tc##1{\textcolor[rgb]{0.00,0.63,0.00}{##1}}}
\expandafter\def\csname PY@tok@gr\endcsname{\def\PY@tc##1{\textcolor[rgb]{1.00,0.00,0.00}{##1}}}
\expandafter\def\csname PY@tok@ge\endcsname{\let\PY@it=\textit}
\expandafter\def\csname PY@tok@gs\endcsname{\let\PY@bf=\textbf}
\expandafter\def\csname PY@tok@gp\endcsname{\let\PY@bf=\textbf\def\PY@tc##1{\textcolor[rgb]{0.00,0.00,0.50}{##1}}}
\expandafter\def\csname PY@tok@go\endcsname{\def\PY@tc##1{\textcolor[rgb]{0.53,0.53,0.53}{##1}}}
\expandafter\def\csname PY@tok@gt\endcsname{\def\PY@tc##1{\textcolor[rgb]{0.00,0.27,0.87}{##1}}}
\expandafter\def\csname PY@tok@err\endcsname{\def\PY@bc##1{\setlength{\fboxsep}{0pt}\fcolorbox[rgb]{1.00,0.00,0.00}{1,1,1}{\strut ##1}}}
\expandafter\def\csname PY@tok@kc\endcsname{\let\PY@bf=\textbf\def\PY@tc##1{\textcolor[rgb]{0.00,0.50,0.00}{##1}}}
\expandafter\def\csname PY@tok@kd\endcsname{\let\PY@bf=\textbf\def\PY@tc##1{\textcolor[rgb]{0.00,0.50,0.00}{##1}}}
\expandafter\def\csname PY@tok@kn\endcsname{\let\PY@bf=\textbf\def\PY@tc##1{\textcolor[rgb]{0.00,0.50,0.00}{##1}}}
\expandafter\def\csname PY@tok@kr\endcsname{\let\PY@bf=\textbf\def\PY@tc##1{\textcolor[rgb]{0.00,0.50,0.00}{##1}}}
\expandafter\def\csname PY@tok@bp\endcsname{\def\PY@tc##1{\textcolor[rgb]{0.00,0.50,0.00}{##1}}}
\expandafter\def\csname PY@tok@fm\endcsname{\def\PY@tc##1{\textcolor[rgb]{0.00,0.00,1.00}{##1}}}
\expandafter\def\csname PY@tok@vc\endcsname{\def\PY@tc##1{\textcolor[rgb]{0.10,0.09,0.49}{##1}}}
\expandafter\def\csname PY@tok@vg\endcsname{\def\PY@tc##1{\textcolor[rgb]{0.10,0.09,0.49}{##1}}}
\expandafter\def\csname PY@tok@vi\endcsname{\def\PY@tc##1{\textcolor[rgb]{0.10,0.09,0.49}{##1}}}
\expandafter\def\csname PY@tok@vm\endcsname{\def\PY@tc##1{\textcolor[rgb]{0.10,0.09,0.49}{##1}}}
\expandafter\def\csname PY@tok@sa\endcsname{\def\PY@tc##1{\textcolor[rgb]{0.73,0.13,0.13}{##1}}}
\expandafter\def\csname PY@tok@sb\endcsname{\def\PY@tc##1{\textcolor[rgb]{0.73,0.13,0.13}{##1}}}
\expandafter\def\csname PY@tok@sc\endcsname{\def\PY@tc##1{\textcolor[rgb]{0.73,0.13,0.13}{##1}}}
\expandafter\def\csname PY@tok@dl\endcsname{\def\PY@tc##1{\textcolor[rgb]{0.73,0.13,0.13}{##1}}}
\expandafter\def\csname PY@tok@s2\endcsname{\def\PY@tc##1{\textcolor[rgb]{0.73,0.13,0.13}{##1}}}
\expandafter\def\csname PY@tok@sh\endcsname{\def\PY@tc##1{\textcolor[rgb]{0.73,0.13,0.13}{##1}}}
\expandafter\def\csname PY@tok@s1\endcsname{\def\PY@tc##1{\textcolor[rgb]{0.73,0.13,0.13}{##1}}}
\expandafter\def\csname PY@tok@mb\endcsname{\def\PY@tc##1{\textcolor[rgb]{0.40,0.40,0.40}{##1}}}
\expandafter\def\csname PY@tok@mf\endcsname{\def\PY@tc##1{\textcolor[rgb]{0.40,0.40,0.40}{##1}}}
\expandafter\def\csname PY@tok@mh\endcsname{\def\PY@tc##1{\textcolor[rgb]{0.40,0.40,0.40}{##1}}}
\expandafter\def\csname PY@tok@mi\endcsname{\def\PY@tc##1{\textcolor[rgb]{0.40,0.40,0.40}{##1}}}
\expandafter\def\csname PY@tok@il\endcsname{\def\PY@tc##1{\textcolor[rgb]{0.40,0.40,0.40}{##1}}}
\expandafter\def\csname PY@tok@mo\endcsname{\def\PY@tc##1{\textcolor[rgb]{0.40,0.40,0.40}{##1}}}
\expandafter\def\csname PY@tok@ch\endcsname{\let\PY@it=\textit\def\PY@tc##1{\textcolor[rgb]{0.25,0.50,0.50}{##1}}}
\expandafter\def\csname PY@tok@cm\endcsname{\let\PY@it=\textit\def\PY@tc##1{\textcolor[rgb]{0.25,0.50,0.50}{##1}}}
\expandafter\def\csname PY@tok@cpf\endcsname{\let\PY@it=\textit\def\PY@tc##1{\textcolor[rgb]{0.25,0.50,0.50}{##1}}}
\expandafter\def\csname PY@tok@c1\endcsname{\let\PY@it=\textit\def\PY@tc##1{\textcolor[rgb]{0.25,0.50,0.50}{##1}}}
\expandafter\def\csname PY@tok@cs\endcsname{\let\PY@it=\textit\def\PY@tc##1{\textcolor[rgb]{0.25,0.50,0.50}{##1}}}

\def\PYZbs{\char`\\}
\def\PYZus{\char`\_}
\def\PYZob{\char`\{}
\def\PYZcb{\char`\}}
\def\PYZca{\char`\^}
\def\PYZam{\char`\&}
\def\PYZlt{\char`\<}
\def\PYZgt{\char`\>}
\def\PYZsh{\char`\#}
\def\PYZpc{\char`\%}
\def\PYZdl{\char`\$}
\def\PYZhy{\char`\-}
\def\PYZsq{\char`\'}
\def\PYZdq{\char`\"}
\def\PYZti{\char`\~}
% for compatibility with earlier versions
\def\PYZat{@}
\def\PYZlb{[}
\def\PYZrb{]}
\makeatother


    % Exact colors from NB
    \definecolor{incolor}{rgb}{0.0, 0.0, 0.5}
    \definecolor{outcolor}{rgb}{0.545, 0.0, 0.0}



    
    % Prevent overflowing lines due to hard-to-break entities
    \sloppy 
    % Setup hyperref package
    \hypersetup{
      breaklinks=true,  % so long urls are correctly broken across lines
      colorlinks=true,
      urlcolor=urlcolor,
      linkcolor=linkcolor,
      citecolor=citecolor,
      }
    % Slightly bigger margins than the latex defaults
    
    \geometry{verbose,tmargin=1in,bmargin=1in,lmargin=1in,rmargin=1in}
    
    

    \begin{document}
    
    
    \maketitle
    
    

    
    \hypertarget{modeling-and-simulation-in-python}{%
\section{Modeling and Simulation in
Python}\label{modeling-and-simulation-in-python}}

Chapter 4

Copyright 2017 Allen Downey

License: \href{https://creativecommons.org/licenses/by/4.0}{Creative
Commons Attribution 4.0 International}

    \begin{Verbatim}[commandchars=\\\{\}]
{\color{incolor}In [{\color{incolor}2}]:} \PY{c+c1}{\PYZsh{} Configure Jupyter so figures appear in the notebook}
        \PY{o}{\PYZpc{}}\PY{k}{matplotlib} inline
        
        \PY{c+c1}{\PYZsh{} Configure Jupyter to display the assigned value after an assignment}
        \PY{o}{\PYZpc{}}\PY{k}{config} InteractiveShell.ast\PYZus{}node\PYZus{}interactivity=\PYZsq{}last\PYZus{}expr\PYZus{}or\PYZus{}assign\PYZsq{}
        
        \PY{c+c1}{\PYZsh{} import functions from the modsim library}
        \PY{k+kn}{from} \PY{n+nn}{modsim} \PY{k}{import} \PY{o}{*}
\end{Verbatim}


    \hypertarget{returning-values}{%
\subsection{Returning values}\label{returning-values}}

    Here's a simple function that returns a value:

    \begin{Verbatim}[commandchars=\\\{\}]
{\color{incolor}In [{\color{incolor}3}]:} \PY{k}{def} \PY{n+nf}{add\PYZus{}five}\PY{p}{(}\PY{n}{x}\PY{p}{)}\PY{p}{:}
            \PY{k}{return} \PY{n}{x} \PY{o}{+} \PY{l+m+mi}{5}
\end{Verbatim}


    And here's how we call it.

    \begin{Verbatim}[commandchars=\\\{\}]
{\color{incolor}In [{\color{incolor}4}]:} \PY{n}{y} \PY{o}{=} \PY{n}{add\PYZus{}five}\PY{p}{(}\PY{l+m+mi}{3}\PY{p}{)}
\end{Verbatim}


\begin{Verbatim}[commandchars=\\\{\}]
{\color{outcolor}Out[{\color{outcolor}4}]:} 8
\end{Verbatim}
            
    If you run a function on the last line of a cell, Jupyter displays the
result:

    \begin{Verbatim}[commandchars=\\\{\}]
{\color{incolor}In [{\color{incolor}5}]:} \PY{n}{add\PYZus{}five}\PY{p}{(}\PY{l+m+mi}{5}\PY{p}{)}
\end{Verbatim}


\begin{Verbatim}[commandchars=\\\{\}]
{\color{outcolor}Out[{\color{outcolor}5}]:} 10
\end{Verbatim}
            
    But that can be a bad habit, because usually if you call a function and
don't assign the result in a variable, the result gets discarded.

In the following example, Jupyter shows the second result, but the first
result just disappears.

    \begin{Verbatim}[commandchars=\\\{\}]
{\color{incolor}In [{\color{incolor}6}]:} \PY{n}{add\PYZus{}five}\PY{p}{(}\PY{l+m+mi}{3}\PY{p}{)}
        \PY{n}{add\PYZus{}five}\PY{p}{(}\PY{l+m+mi}{5}\PY{p}{)}
\end{Verbatim}


\begin{Verbatim}[commandchars=\\\{\}]
{\color{outcolor}Out[{\color{outcolor}6}]:} 10
\end{Verbatim}
            
    When you call a function that returns a variable, it is generally a good
idea to assign the result to a variable.

    \begin{Verbatim}[commandchars=\\\{\}]
{\color{incolor}In [{\color{incolor}7}]:} \PY{n}{y1} \PY{o}{=} \PY{n}{add\PYZus{}five}\PY{p}{(}\PY{l+m+mi}{3}\PY{p}{)}
        \PY{n}{y2} \PY{o}{=} \PY{n}{add\PYZus{}five}\PY{p}{(}\PY{l+m+mi}{5}\PY{p}{)}
        
        \PY{n+nb}{print}\PY{p}{(}\PY{n}{y1}\PY{p}{,} \PY{n}{y2}\PY{p}{)}
\end{Verbatim}


    \begin{Verbatim}[commandchars=\\\{\}]
8 10

    \end{Verbatim}

    \textbf{Exercise:} Write a function called \texttt{make\_state} that
creates a \texttt{State} object with the state variables
\texttt{olin=10} and \texttt{wellesley=2}, and then returns the new
\texttt{State} object.

Write a line of code that calls \texttt{make\_state} and assigns the
result to a variable named \texttt{init}.

    \begin{Verbatim}[commandchars=\\\{\}]
{\color{incolor}In [{\color{incolor}8}]:} \PY{c+c1}{\PYZsh{} Solution goes here}
\end{Verbatim}


    \begin{Verbatim}[commandchars=\\\{\}]
{\color{incolor}In [{\color{incolor}9}]:} \PY{c+c1}{\PYZsh{} Solution goes here}
\end{Verbatim}


    \hypertarget{running-simulations}{%
\subsection{Running simulations}\label{running-simulations}}

    Here's the code from the previous notebook.

    \begin{Verbatim}[commandchars=\\\{\}]
{\color{incolor}In [{\color{incolor}10}]:} \PY{k}{def} \PY{n+nf}{step}\PY{p}{(}\PY{n}{state}\PY{p}{,} \PY{n}{p1}\PY{p}{,} \PY{n}{p2}\PY{p}{)}\PY{p}{:}
             \PY{l+s+sd}{\PYZdq{}\PYZdq{}\PYZdq{}Simulate one minute of time.}
         \PY{l+s+sd}{    }
         \PY{l+s+sd}{    state: bikeshare State object}
         \PY{l+s+sd}{    p1: probability of an Olin\PYZhy{}\PYZgt{}Wellesley customer arrival}
         \PY{l+s+sd}{    p2: probability of a Wellesley\PYZhy{}\PYZgt{}Olin customer arrival}
         \PY{l+s+sd}{    \PYZdq{}\PYZdq{}\PYZdq{}}
             \PY{k}{if} \PY{n}{flip}\PY{p}{(}\PY{n}{p1}\PY{p}{)}\PY{p}{:}
                 \PY{n}{bike\PYZus{}to\PYZus{}wellesley}\PY{p}{(}\PY{n}{state}\PY{p}{)}
             
             \PY{k}{if} \PY{n}{flip}\PY{p}{(}\PY{n}{p2}\PY{p}{)}\PY{p}{:}
                 \PY{n}{bike\PYZus{}to\PYZus{}olin}\PY{p}{(}\PY{n}{state}\PY{p}{)}
                 
         \PY{k}{def} \PY{n+nf}{bike\PYZus{}to\PYZus{}wellesley}\PY{p}{(}\PY{n}{state}\PY{p}{)}\PY{p}{:}
             \PY{l+s+sd}{\PYZdq{}\PYZdq{}\PYZdq{}Move one bike from Olin to Wellesley.}
         \PY{l+s+sd}{    }
         \PY{l+s+sd}{    state: bikeshare State object}
         \PY{l+s+sd}{    \PYZdq{}\PYZdq{}\PYZdq{}}
             \PY{k}{if} \PY{n}{state}\PY{o}{.}\PY{n}{olin} \PY{o}{==} \PY{l+m+mi}{0}\PY{p}{:}
                 \PY{n}{state}\PY{o}{.}\PY{n}{olin\PYZus{}empty} \PY{o}{+}\PY{o}{=} \PY{l+m+mi}{1}
                 \PY{k}{return}
             \PY{n}{state}\PY{o}{.}\PY{n}{olin} \PY{o}{\PYZhy{}}\PY{o}{=} \PY{l+m+mi}{1}
             \PY{n}{state}\PY{o}{.}\PY{n}{wellesley} \PY{o}{+}\PY{o}{=} \PY{l+m+mi}{1}
             
         \PY{k}{def} \PY{n+nf}{bike\PYZus{}to\PYZus{}olin}\PY{p}{(}\PY{n}{state}\PY{p}{)}\PY{p}{:}
             \PY{l+s+sd}{\PYZdq{}\PYZdq{}\PYZdq{}Move one bike from Wellesley to Olin.}
         \PY{l+s+sd}{    }
         \PY{l+s+sd}{    state: bikeshare State object}
         \PY{l+s+sd}{    \PYZdq{}\PYZdq{}\PYZdq{}}
             \PY{k}{if} \PY{n}{state}\PY{o}{.}\PY{n}{wellesley} \PY{o}{==} \PY{l+m+mi}{0}\PY{p}{:}
                 \PY{n}{state}\PY{o}{.}\PY{n}{wellesley\PYZus{}empty} \PY{o}{+}\PY{o}{=} \PY{l+m+mi}{1}
                 \PY{k}{return}
             \PY{n}{state}\PY{o}{.}\PY{n}{wellesley} \PY{o}{\PYZhy{}}\PY{o}{=} \PY{l+m+mi}{1}
             \PY{n}{state}\PY{o}{.}\PY{n}{olin} \PY{o}{+}\PY{o}{=} \PY{l+m+mi}{1}
             
         \PY{k}{def} \PY{n+nf}{decorate\PYZus{}bikeshare}\PY{p}{(}\PY{p}{)}\PY{p}{:}
             \PY{l+s+sd}{\PYZdq{}\PYZdq{}\PYZdq{}Add a title and label the axes.\PYZdq{}\PYZdq{}\PYZdq{}}
             \PY{n}{decorate}\PY{p}{(}\PY{n}{title}\PY{o}{=}\PY{l+s+s1}{\PYZsq{}}\PY{l+s+s1}{Olin\PYZhy{}Wellesley Bikeshare}\PY{l+s+s1}{\PYZsq{}}\PY{p}{,}
                      \PY{n}{xlabel}\PY{o}{=}\PY{l+s+s1}{\PYZsq{}}\PY{l+s+s1}{Time step (min)}\PY{l+s+s1}{\PYZsq{}}\PY{p}{,} 
                      \PY{n}{ylabel}\PY{o}{=}\PY{l+s+s1}{\PYZsq{}}\PY{l+s+s1}{Number of bikes}\PY{l+s+s1}{\PYZsq{}}\PY{p}{)}
\end{Verbatim}


    Here's a modified version of \texttt{run\_simulation} that creates a
\texttt{State} object, runs the simulation, and returns the
\texttt{State} object.

    \begin{Verbatim}[commandchars=\\\{\}]
{\color{incolor}In [{\color{incolor}11}]:} \PY{k}{def} \PY{n+nf}{run\PYZus{}simulation}\PY{p}{(}\PY{n}{p1}\PY{p}{,} \PY{n}{p2}\PY{p}{,} \PY{n}{num\PYZus{}steps}\PY{p}{)}\PY{p}{:}
             \PY{l+s+sd}{\PYZdq{}\PYZdq{}\PYZdq{}Simulate the given number of time steps.}
         \PY{l+s+sd}{    }
         \PY{l+s+sd}{    p1: probability of an Olin\PYZhy{}\PYZgt{}Wellesley customer arrival}
         \PY{l+s+sd}{    p2: probability of a Wellesley\PYZhy{}\PYZgt{}Olin customer arrival}
         \PY{l+s+sd}{    num\PYZus{}steps: number of time steps}
         \PY{l+s+sd}{    \PYZdq{}\PYZdq{}\PYZdq{}}
             \PY{n}{state} \PY{o}{=} \PY{n}{State}\PY{p}{(}\PY{n}{olin}\PY{o}{=}\PY{l+m+mi}{10}\PY{p}{,} \PY{n}{wellesley}\PY{o}{=}\PY{l+m+mi}{2}\PY{p}{,} 
                           \PY{n}{olin\PYZus{}empty}\PY{o}{=}\PY{l+m+mi}{0}\PY{p}{,} \PY{n}{wellesley\PYZus{}empty}\PY{o}{=}\PY{l+m+mi}{0}\PY{p}{)}
                             
             \PY{k}{for} \PY{n}{i} \PY{o+ow}{in} \PY{n+nb}{range}\PY{p}{(}\PY{n}{num\PYZus{}steps}\PY{p}{)}\PY{p}{:}
                 \PY{n}{step}\PY{p}{(}\PY{n}{state}\PY{p}{,} \PY{n}{p1}\PY{p}{,} \PY{n}{p2}\PY{p}{)}
                 
             \PY{k}{return} \PY{n}{state}
\end{Verbatim}


    Now \texttt{run\_simulation} doesn't plot anything:

    \begin{Verbatim}[commandchars=\\\{\}]
{\color{incolor}In [{\color{incolor}12}]:} \PY{n}{state} \PY{o}{=} \PY{n}{run\PYZus{}simulation}\PY{p}{(}\PY{l+m+mf}{0.4}\PY{p}{,} \PY{l+m+mf}{0.2}\PY{p}{,} \PY{l+m+mi}{60}\PY{p}{)}
\end{Verbatim}


\begin{Verbatim}[commandchars=\\\{\}]
{\color{outcolor}Out[{\color{outcolor}12}]:} olin                0
         wellesley          12
         olin\_empty          8
         wellesley\_empty     0
         dtype: int64
\end{Verbatim}
            
    But after the simulation, we can read the metrics from the
\texttt{State} object.

    \begin{Verbatim}[commandchars=\\\{\}]
{\color{incolor}In [{\color{incolor}13}]:} \PY{n}{state}\PY{o}{.}\PY{n}{olin\PYZus{}empty}
\end{Verbatim}


\begin{Verbatim}[commandchars=\\\{\}]
{\color{outcolor}Out[{\color{outcolor}13}]:} 8
\end{Verbatim}
            
    Now we can run simulations with different values for the parameters.
When \texttt{p1} is small, we probably don't run out of bikes at Olin.

    \begin{Verbatim}[commandchars=\\\{\}]
{\color{incolor}In [{\color{incolor}14}]:} \PY{n}{state} \PY{o}{=} \PY{n}{run\PYZus{}simulation}\PY{p}{(}\PY{l+m+mf}{0.2}\PY{p}{,} \PY{l+m+mf}{0.2}\PY{p}{,} \PY{l+m+mi}{60}\PY{p}{)}
         \PY{n}{state}\PY{o}{.}\PY{n}{olin\PYZus{}empty}
\end{Verbatim}


\begin{Verbatim}[commandchars=\\\{\}]
{\color{outcolor}Out[{\color{outcolor}14}]:} 0
\end{Verbatim}
            
    When \texttt{p1} is large, we probably do.

    \begin{Verbatim}[commandchars=\\\{\}]
{\color{incolor}In [{\color{incolor}15}]:} \PY{n}{state} \PY{o}{=} \PY{n}{run\PYZus{}simulation}\PY{p}{(}\PY{l+m+mf}{0.6}\PY{p}{,} \PY{l+m+mf}{0.2}\PY{p}{,} \PY{l+m+mi}{60}\PY{p}{)}
         \PY{n}{state}\PY{o}{.}\PY{n}{olin\PYZus{}empty}
\end{Verbatim}


\begin{Verbatim}[commandchars=\\\{\}]
{\color{outcolor}Out[{\color{outcolor}15}]:} 20
\end{Verbatim}
            
    \hypertarget{more-for-loops}{%
\subsection{More for loops}\label{more-for-loops}}

    \texttt{linspace} creates a NumPy array of equally spaced numbers.

    \begin{Verbatim}[commandchars=\\\{\}]
{\color{incolor}In [{\color{incolor}16}]:} \PY{n}{p1\PYZus{}array} \PY{o}{=} \PY{n}{linspace}\PY{p}{(}\PY{l+m+mi}{0}\PY{p}{,} \PY{l+m+mi}{1}\PY{p}{,} \PY{l+m+mi}{5}\PY{p}{)}
\end{Verbatim}


\begin{Verbatim}[commandchars=\\\{\}]
{\color{outcolor}Out[{\color{outcolor}16}]:} array([0.  , 0.25, 0.5 , 0.75, 1.  ])
\end{Verbatim}
            
    We can use an array in a \texttt{for} loop, like this:

    \begin{Verbatim}[commandchars=\\\{\}]
{\color{incolor}In [{\color{incolor}17}]:} \PY{k}{for} \PY{n}{p1} \PY{o+ow}{in} \PY{n}{p1\PYZus{}array}\PY{p}{:}
             \PY{n+nb}{print}\PY{p}{(}\PY{n}{p1}\PY{p}{)}
\end{Verbatim}


    \begin{Verbatim}[commandchars=\\\{\}]
0.0
0.25
0.5
0.75
1.0

    \end{Verbatim}

    This will come in handy in the next section.

\texttt{linspace} is defined in \texttt{modsim.py}. You can get the
documentation using \texttt{help}.

    \begin{Verbatim}[commandchars=\\\{\}]
{\color{incolor}In [{\color{incolor}18}]:} \PY{n}{help}\PY{p}{(}\PY{n}{linspace}\PY{p}{)}
\end{Verbatim}


    \begin{Verbatim}[commandchars=\\\{\}]
Help on function linspace in module modsim:

linspace(start, stop, num=50, **options)
    Returns an array of evenly-spaced values in the interval [start, stop].
    
    start: first value
    stop: last value
    num: number of values
    
    Also accepts the same keyword arguments as np.linspace.  See
    https://docs.scipy.org/doc/numpy/reference/generated/numpy.linspace.html
    
    returns: array or Quantity


    \end{Verbatim}

    \texttt{linspace} is based on a NumPy function with the same name.
\href{https://docs.scipy.org/doc/numpy/reference/generated/numpy.linspace.html}{Click
here} to read more about how to use it.

    \textbf{Exercise:} Use \texttt{linspace} to make an array of 10 equally
spaced numbers from 1 to 10 (including both).

    \begin{Verbatim}[commandchars=\\\{\}]
{\color{incolor}In [{\color{incolor}19}]:} \PY{c+c1}{\PYZsh{} Solution goes here}
\end{Verbatim}


    \textbf{Exercise:} The \texttt{modsim} library provides a related
function called \texttt{linrange}. You can view the documentation by
running the following cell:

    \begin{Verbatim}[commandchars=\\\{\}]
{\color{incolor}In [{\color{incolor}20}]:} \PY{n}{help}\PY{p}{(}\PY{n}{linrange}\PY{p}{)}
\end{Verbatim}


    \begin{Verbatim}[commandchars=\\\{\}]
Help on function linrange in module modsim:

linrange(start=0, stop=None, step=1, **options)
    Returns an array of evenly-spaced values in the interval [start, stop].
    
    This function works best if the space between start and stop
    is divisible by step; otherwise the results might be surprising.
    
    By default, the last value in the array is `stop-step`
    (at least approximately).
    If you provide the keyword argument `endpoint=True`,
    the last value in the array is `stop`.
    
    start: first value
    stop: last value
    step: space between values
    
    Also accepts the same keyword arguments as np.linspace.  See
    https://docs.scipy.org/doc/numpy/reference/generated/numpy.linspace.html
    
    returns: array or Quantity


    \end{Verbatim}

    Use \texttt{linrange} to make an array of numbers from 1 to 11 with a
step size of 2.

    \begin{Verbatim}[commandchars=\\\{\}]
{\color{incolor}In [{\color{incolor}21}]:} \PY{c+c1}{\PYZsh{} Solution goes here}
\end{Verbatim}


    \hypertarget{sweeping-parameters}{%
\subsection{Sweeping parameters}\label{sweeping-parameters}}

    \texttt{p1\_array} contains a range of values for \texttt{p1}.

    \begin{Verbatim}[commandchars=\\\{\}]
{\color{incolor}In [{\color{incolor}22}]:} \PY{n}{p2} \PY{o}{=} \PY{l+m+mf}{0.2}
         \PY{n}{num\PYZus{}steps} \PY{o}{=} \PY{l+m+mi}{60}
         \PY{n}{p1\PYZus{}array} \PY{o}{=} \PY{n}{linspace}\PY{p}{(}\PY{l+m+mi}{0}\PY{p}{,} \PY{l+m+mi}{1}\PY{p}{,} \PY{l+m+mi}{11}\PY{p}{)}
\end{Verbatim}


\begin{Verbatim}[commandchars=\\\{\}]
{\color{outcolor}Out[{\color{outcolor}22}]:} array([0. , 0.1, 0.2, 0.3, 0.4, 0.5, 0.6, 0.7, 0.8, 0.9, 1. ])
\end{Verbatim}
            
    The following loop runs a simulation for each value of \texttt{p1} in
\texttt{p1\_array}; after each simulation, it prints the number of
unhappy customers at the Olin station:

    \begin{Verbatim}[commandchars=\\\{\}]
{\color{incolor}In [{\color{incolor}23}]:} \PY{k}{for} \PY{n}{p1} \PY{o+ow}{in} \PY{n}{p1\PYZus{}array}\PY{p}{:}
             \PY{n}{state} \PY{o}{=} \PY{n}{run\PYZus{}simulation}\PY{p}{(}\PY{n}{p1}\PY{p}{,} \PY{n}{p2}\PY{p}{,} \PY{n}{num\PYZus{}steps}\PY{p}{)}
             \PY{n+nb}{print}\PY{p}{(}\PY{n}{p1}\PY{p}{,} \PY{n}{state}\PY{o}{.}\PY{n}{olin\PYZus{}empty}\PY{p}{)}
\end{Verbatim}


    \begin{Verbatim}[commandchars=\\\{\}]
0.0 0
0.1 0
0.2 0
0.30000000000000004 0
0.4 0
0.5 13
0.6000000000000001 19
0.7000000000000001 19
0.8 30
0.9 36
1.0 39

    \end{Verbatim}

    Now we can do the same thing, but storing the results in a
\texttt{SweepSeries} instead of printing them.

    \begin{Verbatim}[commandchars=\\\{\}]
{\color{incolor}In [{\color{incolor}24}]:} \PY{n}{sweep} \PY{o}{=} \PY{n}{SweepSeries}\PY{p}{(}\PY{p}{)}
         
         \PY{k}{for} \PY{n}{p1} \PY{o+ow}{in} \PY{n}{p1\PYZus{}array}\PY{p}{:}
             \PY{n}{state} \PY{o}{=} \PY{n}{run\PYZus{}simulation}\PY{p}{(}\PY{n}{p1}\PY{p}{,} \PY{n}{p2}\PY{p}{,} \PY{n}{num\PYZus{}steps}\PY{p}{)}
             \PY{n}{sweep}\PY{p}{[}\PY{n}{p1}\PY{p}{]} \PY{o}{=} \PY{n}{state}\PY{o}{.}\PY{n}{olin\PYZus{}empty}
\end{Verbatim}


    And then we can plot the results.

    \begin{Verbatim}[commandchars=\\\{\}]
{\color{incolor}In [{\color{incolor}25}]:} \PY{n}{plot}\PY{p}{(}\PY{n}{sweep}\PY{p}{,} \PY{n}{label}\PY{o}{=}\PY{l+s+s1}{\PYZsq{}}\PY{l+s+s1}{Olin}\PY{l+s+s1}{\PYZsq{}}\PY{p}{)}
         
         \PY{n}{decorate}\PY{p}{(}\PY{n}{title}\PY{o}{=}\PY{l+s+s1}{\PYZsq{}}\PY{l+s+s1}{Olin\PYZhy{}Wellesley Bikeshare}\PY{l+s+s1}{\PYZsq{}}\PY{p}{,}
                  \PY{n}{xlabel}\PY{o}{=}\PY{l+s+s1}{\PYZsq{}}\PY{l+s+s1}{Arrival rate at Olin (p1 in customers/min)}\PY{l+s+s1}{\PYZsq{}}\PY{p}{,} 
                  \PY{n}{ylabel}\PY{o}{=}\PY{l+s+s1}{\PYZsq{}}\PY{l+s+s1}{Number of unhappy customers}\PY{l+s+s1}{\PYZsq{}}\PY{p}{)}
         
         \PY{n}{savefig}\PY{p}{(}\PY{l+s+s1}{\PYZsq{}}\PY{l+s+s1}{figs/chap02\PYZhy{}fig02.pdf}\PY{l+s+s1}{\PYZsq{}}\PY{p}{)}
\end{Verbatim}


    \begin{Verbatim}[commandchars=\\\{\}]
Saving figure to file figs/chap02-fig02.pdf

    \end{Verbatim}

    \begin{center}
    \adjustimage{max size={0.9\linewidth}{0.9\paperheight}}{output_51_1.png}
    \end{center}
    { \hspace*{\fill} \\}
    
    \hypertarget{exercises}{%
\subsection{Exercises}\label{exercises}}

\textbf{Exercise:} Wrap this code in a function named \texttt{sweep\_p1}
that takes an array called \texttt{p1\_array} as a parameter. It should
create a new \texttt{SweepSeries}, run a simulation for each value of
\texttt{p1} in \texttt{p1\_array}, store the results in the
\texttt{SweepSeries}, and return the \texttt{SweepSeries}.

Use your function to plot the number of unhappy customers at Olin as a
function of \texttt{p1}. Label the axes.

    \begin{Verbatim}[commandchars=\\\{\}]
{\color{incolor}In [{\color{incolor}26}]:} \PY{c+c1}{\PYZsh{} Solution goes here}
\end{Verbatim}


    \begin{Verbatim}[commandchars=\\\{\}]
{\color{incolor}In [{\color{incolor}27}]:} \PY{c+c1}{\PYZsh{} Solution goes here}
\end{Verbatim}


    \textbf{Exercise:} Write a function called \texttt{sweep\_p2} that runs
simulations with \texttt{p1=0.5} and a range of values for \texttt{p2}.
It should store the results in a \texttt{SweepSeries} and return the
\texttt{SweepSeries}.

    \begin{Verbatim}[commandchars=\\\{\}]
{\color{incolor}In [{\color{incolor}28}]:} \PY{c+c1}{\PYZsh{} Solution goes here}
\end{Verbatim}


    \begin{Verbatim}[commandchars=\\\{\}]
{\color{incolor}In [{\color{incolor}29}]:} \PY{c+c1}{\PYZsh{} Solution goes here}
\end{Verbatim}


    \hypertarget{optional-exercises}{%
\subsection{Optional exercises}\label{optional-exercises}}

The following two exercises are a little more challenging. If you are
comfortable with what you have learned so far, you should give them a
try. If you feel like you have your hands full, you might want to skip
them for now.

\textbf{Exercise:} Because our simulations are random, the results vary
from one run to another, and the results of a parameter sweep tend to be
noisy. We can get a clearer picture of the relationship between a
parameter and a metric by running multiple simulations with the same
parameter and taking the average of the results.

Write a function called \texttt{run\_multiple\_simulations} that takes
as parameters \texttt{p1}, \texttt{p2}, \texttt{num\_steps}, and
\texttt{num\_runs}.

\texttt{num\_runs} specifies how many times it should call
\texttt{run\_simulation}.

After each run, it should store the total number of unhappy customers
(at Olin or Wellesley) in a \texttt{TimeSeries}. At the end, it should
return the \texttt{TimeSeries}.

Test your function with parameters

\begin{verbatim}
p1 = 0.3
p2 = 0.3
num_steps = 60
num_runs = 10
\end{verbatim}

Display the resulting \texttt{TimeSeries} and use the \texttt{mean}
function provided by the \texttt{TimeSeries} object to compute the
average number of unhappy customers.

    \begin{Verbatim}[commandchars=\\\{\}]
{\color{incolor}In [{\color{incolor}30}]:} \PY{c+c1}{\PYZsh{} Solution goes here}
\end{Verbatim}


    \begin{Verbatim}[commandchars=\\\{\}]
{\color{incolor}In [{\color{incolor}31}]:} \PY{c+c1}{\PYZsh{} Solution goes here}
\end{Verbatim}


    \textbf{Exercise:} Continuting the previous exercise, use
\texttt{run\_multiple\_simulations} to run simulations with a range of
values for \texttt{p1} and

\begin{verbatim}
p2 = 0.3
num_steps = 60
num_runs = 20
\end{verbatim}

Store the results in a \texttt{SweepSeries}, then plot the average
number of unhappy customers as a function of \texttt{p1}. Label the
axes.

What value of \texttt{p1} minimizes the average number of unhappy
customers?

    \begin{Verbatim}[commandchars=\\\{\}]
{\color{incolor}In [{\color{incolor}32}]:} \PY{c+c1}{\PYZsh{} Solution goes here}
\end{Verbatim}


    \begin{Verbatim}[commandchars=\\\{\}]
{\color{incolor}In [{\color{incolor}33}]:} \PY{c+c1}{\PYZsh{} Solution goes here}
\end{Verbatim}



    % Add a bibliography block to the postdoc
    
    
    
    \end{document}
